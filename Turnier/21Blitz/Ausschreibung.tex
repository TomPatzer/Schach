% This file was converted to LaTeX by Writer2LaTeX ver. 1.4
% see http://writer2latex.sourceforge.net for more info
\documentclass[a4paper,parskip=full-,DIV15]{scrartcl}
\usepackage{iftex}
\ifPDFTeX
\usepackage[utf8]{inputenc}  % Deutsche Umlaute direkt eingeben
\usepackage[T1]{fontenc}     % "Saubere" Schriften in der PDF
\usepackage{amsfonts, amssymb, amstext,amsthm}
\else
\ifXeTeX
\usepackage{fontspec}
\usepackage{unicode-math}
\fontspec{DejaVu Serif}
\setmainfont{DejaVu Serif} 
\setsansfont{DejaVu Sans} 
\setmonofont{DejaVu Sans Mono} 
\setmathfont[math-style=TeX]{TeX Gyre DejaVu Math}
\else
\ifLuaTeX
\usepackage{fontspec}
\usepackage{luatextra}
\usepackage{unicode-math}
\fontspec{DejaVu Serif}
\setmainfont{DejaVu Serif} 
\setsansfont{DejaVu Sans} 
\setmonofont{DejaVu Sans Mono} 
\setmathfont[math-style=TeX]{TeX Gyre DejaVu Math}
\fi
\fi 				
\fi
\usepackage[ngerman]{babel}
\usepackage{textcomp}
\usepackage{color}
\usepackage{array}
\usepackage{hhline}
\renewcommand{\arraystretch}{1.5}
\usepackage[colorlinks=true, urlcolor=blue]{hyperref} 

\title{Monatsblitz 2021 /2022 \\Schachfreunde Drensteinfurt e.V.}
\date{}

\begin{document}

\maketitle


\begin{tabular}{p{4.0 cm} p{12.5 cm}}
	\textbf{Temine:}     & Jeweils der letzte Freitag vin September 2021 bis Februar 2022 mit Ausnahme des Dezembers: Also 24.09.21, 29.10.21, 26.11.21, 28.01.22  und 25.02.22 jeweils um 19:15 Uhr  \\
	\textbf{Teilnehmer:} & Mitglieder und Freund*innen der „Schachfreunde Drensteinfurt“\\
	\textbf{Modus:}      & 5 Minuten pro Spieler*in /Partie nach FIDE-Blitzschachregeln. Ausnahme: Ein ungültiger Zug kann reklamiert werden und verliert die Partie.  \\
	                     &Bei weniger als 6 Spieler*innen wird doppelründig gespielt. Zwischen 7  und 11 Spieler*innen findet das Turnier als Rundenturnier statt. Bei mehr als 11 Spieler*innen findet das Turnier im Schweizersystem mit 10 Runden statt. \\
   \textbf{Punkte für die Gesamtwertung:}     & Für die Gesamtwertung werden die Tagesplatzierungen herangezogen. Sollten weniger als 11 Spieler*innen gespiel haben, bekommt jede Spieler*in die folgende Anzahl von Punkten. Man zieht die Platzierung von der Anzahl der Teilnehmer + 1 ab.\\
   & Beispiel: Bei 7 Spieler*innen bekommt der Zweitplatzierte $ 7 + 1 -2 $ Punkte. Sollten mehr als 10 Personen an einem Abend mitgespielt haben gilt Formel 11 - Platzierung. Für das Gesamtergebnis gelten dann die 4 besten Tagesergebnisse. \\
	\textbf{Preise:}     & Der/die Erstplatzierte erhält eine Urkunde.\\
	\textbf{Spiellokal:} & Kulturbahnhof Drensteinfurt, 1. Etage Raum 4 oder 5\\
\end{tabular}


\bigskip

Anmeldungen bis spätestens 25.09.2021 – 19.15 Uhr am Vereinsabend oder beim Turnierleiter \\ \href{mailto:turnierleiter@schachfreunde-drensteinfurt.de}{turnierleiter@schachfreunde-drensteinfurt.de}



\end{document}

