% This file was converted to LaTeX by Writer2LaTeX ver. 1.4
% see http://writer2latex.sourceforge.net for more info
\documentclass[a4paper,parskip=full-,DIV18]{scrartcl}
\usepackage{iftex}
\ifPDFTeX
\usepackage[utf8]{inputenc}  % Deutsche Umlaute direkt eingeben
\usepackage[T1]{fontenc}     % "Saubere" Schriften in der PDF
\usepackage{amsfonts, amssymb, amstext,amsthm}
\else
\ifXeTeX
\usepackage{fontspec}
\usepackage{unicode-math}
\fontspec{DejaVu Serif}
\setmainfont{DejaVu Serif} 
\setsansfont{DejaVu Sans} 
\setmonofont{DejaVu Sans Mono} 
\setmathfont[math-style=TeX]{TeX Gyre DejaVu Math}
\else
\ifLuaTeX
\usepackage{fontspec}
\usepackage{luatextra}
\usepackage{unicode-math}
\fontspec{DejaVu Serif}
\setmainfont{DejaVu Serif} 
\setsansfont{DejaVu Sans} 
\setmonofont{DejaVu Sans Mono} 
\setmathfont[math-style=TeX]{TeX Gyre DejaVu Math}
\fi
\fi 				
\fi
\usepackage[ngerman]{babel}
\usepackage{textcomp}
\usepackage{color}
\usepackage{array}
\usepackage{hhline}
\renewcommand{\arraystretch}{1.5}
\usepackage[colorlinks=true, urlcolor=blue]{hyperref} 

\title{Grand Prix Schnellschach 2021 \\Schachfreunde Drensteinfurt e.V.}
\date{}

\begin{document}

\maketitle


\begin{tabular}{p{3 cm} p{13.5 cm}}
	\textbf{Turniermodus:} & 5 Doppelrunden, wobei nur die acht besten Ergebnis in die Gesamtwertung eingehen. An jedem Spieltag werden die Partien neu ausgelost. Ab 10 Spieler*innen wird in zwei Gruppen ausgelost. In der ersten Hälfte befinden die Bestplazierten im Laufe des Turniers (bei Gleichstand zählt die DWZ). Die anderen Spieler*innen bilden die zweite Gruppe. Bei ungerader Anzahl der anwesenden Spieler*innen befindet sich in der ersten Gruppe eine Spieler*in mehr. Zu Beginn des Spieltags werden jeweils in den Gruppen beide Partien des Spieltags gelost. Falls sich in der ersten Gruppe ein ungerade Zahl von Spieler*innen befindet, wird eine Partie mit einer Spieler*in der zweiten Gruppe gelost. Etwaige Freilose tauchen nur in der zweiten Gruppe auf. Zwei Freilose für eine Spieler*in an einem Spielabend sind durch das Losverfahren zu vermeiden. Nach dem letzten Spieltag nur die 8 besten Ergebnisse gewertet. Ein Sieg wird mit 3 Punkten und ein Remis mit einem Punkt gewertet. Bei Gleichstand am letzten Spieltag entscheidet der direkte Vergleich. Führt dies nicht zu einer Entscheidung, gibt es mehrere erste Plätze!      \\
	\textbf{Termine:}      & 10.09.2021 19:00 Uhr \\
	                       & 08.10.2021 19:15 Uhr \\
	                       & 12.11.2021 19:15 Uhr \\
	                       & 10.12.2021 19:15 Uhr \\
	                       & 14.01.2022 19:15 Uhr \\
	\textbf{Teilnehmer:}   & Mitglieder und Freund*innen der „Schachfreunde Drensteinfurt“  \\
	\textbf{Bedenkzeit:}   & 25 Minuten pro Spieler*in /Partie plus 10 Sekunden Zeitaufschlag pro Zug! Es herrscht keine Notationspflicht. \\
	\textbf{Wertung:}      & Es erfolgt \textbf{keine} DWZ Auswertung.                                                  \\
	\textbf{Preise:}       & Der/die Erstplatzierte erhält eine Urkunde. \\
	\textbf{Spiellokal:}   & Kulturbahnhof Drensteinfurt, 1. Etage Raum 4 oder 5
\end{tabular}


\bigskip

Anmeldungen bis spätestens 11.09.2021 – 19.15 Uhr am Vereinsabend oder beim Turnierleiter \\ \href{mailto:turnierleiter@schachfreunde-drensteinfurt.de}{turnierleiter@schachfreunde-drensteinfurt.de}



\end{document}
